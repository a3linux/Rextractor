\documentclass[11pt,a4paper]{article}

% basic packages
\usepackage{float}
\usepackage{fullpage}
\usepackage{polski}
\usepackage{graphicx}
\usepackage[utf8x]{inputenc}

% bibliography and links
\usepackage{url}
\usepackage{cite}
\def\UrlBreaks{\do\/\do-}
\usepackage[hidelinks]{hyperref}

% listings
\usepackage{listings}
\usepackage{color}
\definecolor{dkgreen}{rgb}{0,0.6,0}
\definecolor{gray}{rgb}{0.5,0.5,0.5}
\definecolor{mauve}{rgb}{0.58,0,0.82}
\lstset{
  basicstyle=\footnotesize,    
  captionpos=b,             
  commentstyle=\color{dkgreen},  
  frame=single,       
  keywordstyle=\color{blue},  
  language=Python,   
  numbers=left,     
  numbersep=7pt,   
  numberstyle=\tiny\color{gray}, 
  rulecolor=\color{black},  
  stringstyle=\color{mauve}, 
  tabsize=2,    
  title=\lstname
}

\begin{document}

\begin{titlepage}
  \begin{center}

    \textsc{\Large Politechnika Warszawska}\\[0.1cm]
    \small Wydział Elektroniki i Technik Informacyjnych
    \vfill

    \textsc{\small Wprowadzenie do Eksploracji Danych Tekstowych w Środowisku WWW }\\[0.1cm]
    \Huge Gromadzenie i przechowywanie przepisów kulinarnych przy użyciu ontologii\\[1.5cm]
    \small Sprawozdanie wstępne\\[2.5cm]

    \vfill

    \begin{minipage}{0.4\textwidth}
      \begin{flushleft} \large
        \emph{Autorzy:}\\[0.1cm]
        Maciej \textsc{Suchecki}\\
        Michał \textsc{Toporowski}\\
        Jacek \textsc{Witkowski}\\
      \end{flushleft}
    \end{minipage}
    \begin{minipage}{0.4\textwidth}
      \begin{flushright} \large
        \emph{Prowadzący:}\\[0.1cm]
        dr~inż.~Piotr \textsc{Andruszkiewicz}\\[1cm]
      \end{flushright}
    \end{minipage}

    \vfill
    {\large \today}

  \end{center}
\end{titlepage}

\section{Treść zadania}
\paragraph{Tytuł} Gromadzenie i przechowywanie przepisów kulinarnych w ustrukturalizowany sposób, przy użyciu ontologii.
\paragraph{Opis}
\begin{itemize}
  \item ekstrakcja informacji ze stron z przepisami (Information Extraction) - metody do wyboru/ustalenia
  \item odwzorowanie wyekstrahowanych elementów na ontologię
  \item wstawienie instancji do ontologii/ew. modyfikacja ontologii (na poziomie pojęć, dodanie nowych pojęć)
  \item wykonywanie zapytań na ontologii
\end{itemize}

\section{Definicja problemu}
\section{Opis rozwiązania}
\section{Implementacja}
\section{Instrukcja obsługi}
Jak uruchomić?
\section{Testy}
\section{Wnioski (ważny punkt dokumentacji)}

% bibliography
%\addcontentsline{toc}{chapter}{Literatura}
%\bibliography{references}{}
%\bibliographystyle{alpha}


\end{document}
